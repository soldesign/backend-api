%%%%%%%%%%%%%%%%%%%%%%%%%%%%%%%%%%%%%%%%%
% Arsclassica Article
% LaTeX Template
% Version 1.1 (10/6/14)
%
% This template has been downloaded from:
% http://www.LaTeXTemplates.com
%
% Original author:
% Lorenzo Pantieri (http://www.lorenzopantieri.net) with extensive modifications by:
% Vel (vel@latextemplates.com)
%
% License:
% CC BY-NC-SA 3.0 (http://creativecommons.org/licenses/by-nc-sa/3.0/)
%
%%%%%%%%%%%%%%%%%%%%%%%%%%%%%%%%%%%%%%%%%

%----------------------------------------------------------------------------------------
%	PACKAGES AND OTHER DOCUMENT CONFIGURATIONS
%----------------------------------------------------------------------------------------

\documentclass[draft=false,
10pt, % Main document font size
a4paper, % Paper type, use 'letterpaper' for US Letter paper
oneside, % One page layout (no page indentation)
%twoside, % Two page layout (page indentation for binding and different headers)
headinclude,footinclude, % Extra spacing for the header and footer
 % Binding correction
]{scrartcl}

\input{structure.tex} % Include the structure.tex file which specified the document structure and layout

\hyphenation{Fortran hy-phen-ation} % Specify custom hyphenation points in words with dashes where you would like hyphenation to occur, or alternatively, don't put any dashes in a word to stop hyphenation altogether

%----------------------------------------------------------------------------------------
%	TITLE AND AUTHOR(S)
%----------------------------------------------------------------------------------------

\title{\normalfont\spacedallcaps{Architecture of the Karana Back End}} % The article title

\author{\spacedlowsmallcaps{Michael G\"otte*}} % The article author(s) - author affiliations need to be specified in the AUTHOR AFFILIATIONS block

\date{\today} % An optional date to appear under the author(s)

%----------------------------------------------------------------------------------------

\begin{document}

%----------------------------------------------------------------------------------------
%	HEADERS
%----------------------------------------------------------------------------------------

\renewcommand{\sectionmark}[1]{\markright{\spacedlowsmallcaps{#1}}} % The header for all pages (oneside) or for even pages (twoside)
%\renewcommand{\subsectionmark}[1]{\markright{\thesubsection~#1}} % Uncomment when using the twoside option - this modifies the header on odd pages
\lehead{\mbox{\llap{\small\thepage\kern1em\color{halfgray} \vline}\color{halfgray}\hspace{0.5em}\rightmark\hfil}} % The header style

\pagestyle{scrheadings} % Enable the headers specified in this block

%----------------------------------------------------------------------------------------
%	TABLE OF CONTENTS & LISTS OF FIGURES AND TABLES
%----------------------------------------------------------------------------------------

\maketitle % Print the title/author/date block

\setcounter{tocdepth}{2} % Set the depth of the table of contents to show sections and subsections only

\tableofcontents % Print the table of contents





%----------------------------------------------------------------------------------------
%	ABSTRACT
%----------------------------------------------------------------------------------------

\section*{Abstract} % This section will not appear in the table of contents due to the star (\section*)

Karana is a project with the aim of developing a cheap, open source monitoring framework. This document describes the design of the Karana Back End (KBE). The KBE is the IT backbone of the hardware monitoring device. The KBE combines a well established open source approach for time series based data collection and data visualization with a small scale management \href{http://www.restapitutorial.com/}{API} framework and an automated deployment scheme. While there is plenty of information on the Internet how to combine \href{https://www.influxdata.com/}{InfluxDB} with \href{http://grafana.org/}{Grafana} we focus in this documentation on the architecture of the management API and the automated deployment. The managment API is written in the Python framework \href{http://www.hug.rest/}{Hug} which is an API extension of the WSGI framework \href{http://falconframework.org/}{Falcon}. The automated deployment is a collection of scripts written with the powerful automation tool \href{https://www.ansible.com}{Ansible}. 
%----------------------------------------------------------------------------------------
%	AUTHOR AFFILIATIONS
%----------------------------------------------------------------------------------------

{\let\thefootnote\relax\footnotetext{* \textit{MicroEnergy International, Berlin, Germany}}}

%{\let\thefootnote\relax\footnotetext{\textsuperscript{1} \textit{Department of Chemistry, University of Examples, London, United Kingdom}}}

%----------------------------------------------------------------------------------------

\newpage % Start the article content on the second page, remove this if you have a longer abstract that goes onto the second page

%----------------------------------------------------------------------------------------
%	INTRODUCTION
%----------------------------------------------------------------------------------------

\section{Introduction}

The KBE is an IT solution for monitoring data. The KBE provides an IT infrastructure suitable for multiple independent projects on one server, but can also be enrolled separately on different servers for each project. 

The infrastructure relies on several virtual machines managed by \href{https://linuxcontainers.org/lxd/}{LXD}. All incoming requests to the server are directed to a load balancer which forwards the requests based on availability to one of two load balancer which distribute the requests to the correct instances of a project. The load balancing is realized with \href{http://www.haproxy.org/}{HAproxy}. An instance is composed of three LXD containers holding an influxDB instance, a Grafana instance and an management API instance. Those doesn't have to be on the same LXD host. Having a plain Linux (tested on Xenial, Ubuntu 16.04) server with only python installed (a dependency for Ansible) the Ansible script will deploy the IT infrastructure visualized in Figure \ref{figure:infrastructrue} almost automatically.

\begin{figure}
	\centering 
	\begin{tikzpicture}
	\draw[dotted] (-2.5,0.5) node {HTTP Requests};	
	\draw[decoration={markings,mark=at position 1 with {\arrow[scale=3]{>}}},
	postaction={decorate},] (-1,0.5) -- (0,0.5);
	\draw (0,0) -- (0,1) -- (2,1) -- (2,0) -- (0,0) ;
	\draw[dotted] (1,0.5) node {Haproxy};	
	\draw (3,1) -- (3,2) -- (5,2) -- (5,1) -- (3,1) ;
	\draw[dotted] (4,1.5) node {Haproxy1};	
	\draw (3,-1) -- (3,0) -- (5,0) -- (5,-1) -- (3,-1) ;
	\draw[dotted] (4,-0.5) node {Haproxy2};	
	\draw (7,0.75) -- (7,1.25) -- (9,1.25) -- (9,0.75) -- (7,0.75) ;	
	\draw[dotted] (8,1) node {InfluxDB};	
	\draw (7,0.25) -- (7,0.75) -- (9,0.75) -- (9,0.25) -- (7,0.25) ;
	\draw[dotted] (8,0.5) node {Grafana};	
	\draw (7,-0.25) -- (7,0.25) -- (9,0.25) -- (9,-0.25) -- (7,-0.25) ;
	\draw[dotted] (8,0) node {API};	
	\draw (6.75,-0.25) -- (6.75,1.25) -- (7,1.25) -- (7,-0.25) -- (6.75,-0.25) ;
	\draw[dotted] (6.875,0.5) node {1};	
	\draw[decoration={markings,mark=at position 1 with {\arrow[scale=3]{>}}},
	postaction={decorate},] (2.1,0.6) -- (2.9,1.4);
	\draw[decoration={markings,mark=at position 1 with {\arrow[scale=3]{>}}},
	postaction={decorate},] (2.1,0.4) -- (2.9,-0.4);
	\draw (5.1,1.4) -- (6,0.5) -- (5.1,-0.4);
	\draw[decoration={markings,mark=at position 1 with {\arrow[scale=3]{>}}},
	postaction={decorate},] (6,0.5) -- (6.75,0.5);
	\draw (7,3.75) -- (7,4.25) -- (9,4.25) -- (9,3.75) -- (7,3.75) ;	
	\draw[dotted] (8,4) node {InfluxDB};	
	\draw (7,3.25) -- (7,3.75) -- (9,3.75) -- (9,3.25) -- (7,3.25) ;
	\draw[dotted] (8,3.5) node {Grafana};	
	\draw (7,3.75) -- (7,3.25) -- (9,3.25) -- (9,2.75) -- (7,2.75) ;
	\draw[dotted] (8,3) node {API};	
	\draw (6.75,2.75) -- (6.75,4.25) -- (7,4.25) -- (7,2.75) -- (6.75,2.75) ;
	\draw[dotted] (6.875,3.5) node {2};	
	\draw[decoration={markings,mark=at position 1 with {\arrow[scale=3]{>}}},
	postaction={decorate},] (6,0.5) -- (6.75,3.5);
	\draw[decoration={markings,mark=at position 1 with {\arrow[scale=3]{>}}},
	postaction={decorate},] (6,0.5) -- (6.75,2);
	\draw[dotted] (7,2) node {.....};

	\end{tikzpicture}
	\caption[KBE IT infrastructure]{The Karana Back End IT infrastructure} % The text in the square bracket is the caption for the list of figures while the text in the curly brackets is the figure caption
	\label{figure:infrastructrue}
\end{figure}

The next sections will explain  the architecture of the IT infrastructure and how to build it with Ansible. After that we will discuss the software design of the management API and how to use the API. 
 
%----------------------------------------------------------------------------------------
%	Deployment
%----------------------------------------------------------------------------------------

\section{Building the Infrastructure}\label{section:infrastructure}
\subsection{LXD and Virtual Machines}


\subsection{HAproxy and Load Balancing}



\subsection{Achieving Higher Availability}


\section{Configuring an Instance}\label{section:instance}
Having everything setup as in Section \ref{section:infrastructure} to create an instance for a project is now fairly simple. The scripts introduced here will install influxDB, Grafana, the API, and a simple frontent on your target server. Once installed, the containers can be reached via the following URLs, note that right now SSL is not implemented:
\begin{itemize}
	\item InfluxDB API: \nolinkurl{http://<project-name>influx.<your-server-domain-name>}
	\item InfluxDB Admin Panel: \nolinkurl{http://<project-name>influxa.<your-server-domain-name>}	
	\item Grafana: \nolinkurl{http://<project-name>grafana.<your-server-domain-name>}
	\item Karana API: \nolinkurl{http://<project-name>api.<your-server-domain-name>}
	\item Frontend: \nolinkurl{http://<project-name>.<your-server-domain-name>}
\end{itemize}
\textbf{Note (again):} All the paths in here are relative to the base folder of this gitlab project backend-deploy as in Section \ref{section:infrastructure}.

\subsection{Configuration}
The file \textit{playbook/tenant\_var.yml} holds the variables which will be used during the execution of the Ansible scripts. An example file looks like in Listings \ref{listings:tenantvars}.
\begin{lstlisting}[caption={Example of Ansible hosts file},label={listings:tenantvars}]
---
pname: <name>
influxip: 10.0.12.101
grafanaip: 10.0.12.102
apiip: 10.0.12.103
domain: <your-server-domain-name>
path_b: ../../backend-api
path_f_r: ../../frontend-react/karana-app-test
influx_conf: influxdb_1_2.conf
\end{lstlisting}
You can give the project a name and enter your target server domain. The IP addresses are addresses in the virtual network on your target server. Make sure that they are still free. The path variables should point to the KBE base folder and to the frontend of choice. Furthermore, influxDB changes the configuration files from version to version. If the file changed again you can put in \textit{files/templates/} and change the reference in the configuration file. 

\subsection{Create an Instance}
Now in the folder \textit{playbook/} run the command
\begin{tcolorbox}
	\$	ansible-playbool instance.yml -u root
\end{tcolorbox}
This 
\begin{itemize}
	\item creates three influx containers (one for grafana, one for the api, one for influxDB)
	\item adds them to the virtual network on the target server with the defined IP addresses in the file \textit{playbook/tenant\_var.yml}
	\item pushes the authorized keys to the containers for remote access
	\item installs python and ssh on the containers
\end{itemize}
So this scripts creates the raw containers to put in the three different instances.

\subsection{Setup Influx Container}
Now run the command to set up the InfluxDB container
\begin{tcolorbox}
	\$	ansible-playbool influx.yml -u root
\end{tcolorbox}

This
\begin{itemize}
	\item install influx in the container with the influx IP
	\item adds the instance to the two parallel HAproxys by turning the first one off and then turning the second one off, the registered containers are stored in a tinyDB file (you might need to install tinyDB with pip) holding all containers, from the tinyDB entries the template \textit{playbook/templates/haproxy.dist.cfg} is configured using Jinja2 syntax
	\item creates and admin user on the influxDB instance and writes the credentials in the file \textit{files/templates/<projectname>\_config.ini}
\end{itemize} 

\subsection{Setup Grafana Container}
Now run the command to set up the InfluxDB container
\begin{tcolorbox}
	\$	ansible-playbool grafana.yml -u root
\end{tcolorbox}

This
\begin{itemize}
	\item installs grafana in the container with the grafana IP
	\item Creates an Admin user for grafana and writes it into  \textit{files/templates/<projectname>\_config.ini}
	\item copy the scripted dashboard  \textit{files/scripted\_dashboards/karanabase.js} to the Grafana instance
	\item adds the instance to the two parallel HAproxys
\end{itemize}

\subsection{Setup API Container}
Now run the command to set up the API container
\begin{tcolorbox}
	\$	ansible-playbool api.yml -u root
\end{tcolorbox}

This
\begin{itemize}
	\item installs virtualenv, python3-pip, build-essential, libssl-dev, libffi-dev, python3-dev, nginx and git on the container with the API IP
	\item copys the code in the path\_b variable of the \textit{playbook/tenant\_var.yml} into the api container
	\item copys the \textit{files/templates/<projectname>\_config.ini} to the right location (thats how the API knows what influxDB and Grfana instance to use)
	\item installs the requirements of the API in a virtual environment
	\item makes a systemd service to host the API with gunicorn and nginx
	\item adds the instance to the two parallel HAproxys
\end{itemize}

\subsection{Setup API Container}
Now run the command to set up the simple Frontend written in react.js container
\begin{tcolorbox}
	\$	ansible-playbool frontend\_react.yml.yml -u root
\end{tcolorbox}

This
\begin{itemize}
	\item copys the static webpage to \textit{/var/www/html/} on the target server
	\item adds the instance to the two parallel HAproxys
\end{itemize}


%----------------------------------------------------------------------------------------
%	Management API
%----------------------------------------------------------------------------------------

\section{Managment API}
\lstset{language=Python}          % Set your language (you can change the language for each code-block optionally)
Simply spoken the management API enables maintaining the KBE without using command line tools. The API defines several end points which react on incoming HTTP requests. The body format of the HTTP requests (for GET, POST, PUT, PATCH, DELETE) is \href{http://json.org/}{JSON}. Subsection \ref{subsection:endpoints} will specify the behavior of these endpoint. These endpoints can be used to write front ends for better user experience. 
The API is written in Python using the \href{http://www.hug.rest/}{Hug} framework.

For the rest of this Section we assume that the reader has cloned the git repository \href{http://gitlab.me-soldesign.com/karana/backend-api.git}{Git:KBE}. This means that references to files are always with respect to the base folder of the project. 

\subsection{Configuration}
Configuring the API is fairly simple if one has the credentials of the associated influxDB and Grafana instances. Ideally the configuration of the API is done by configuring the instance as described in Section \ref{section:instance}. For completeness we will show how to configure the API by hand. 

There are basically two files which need modification for configuration, the \textit{config.ini} file which holds the information for the credentials of the influxDB and the Grafana instance and the \textit{src/configuration.py} file which holds information on what models are used.
%------------------------------------------------

\subsubsection{The \textit{config.ini} File}
An example  \textit{config.ini} looks like in Listing \ref{listing:config.ini}.
\lstset{language=XML}  
\begin{lstlisting}[caption={Example of a \textit{config.ini} file},label={listing:config.ini}]
[influxdb]
host = 10.0.12.101
port = 8083
user = admin
pass = influxpw

[grafana]
host = 10.0.12.102
port = 3000
user = karanaadmin
pass = karanapasswort


[test]
host = localhost
port = 8000
user = test
pass = test
\end{lstlisting}
One needs to specify the host, the port, the user, and a password. 

\subsubsection{The \textit{src/configuration.py} File}

For an example  of a \textit{src/configuration.py} file see the project repository. In general there are two sections, a resource definition section and a tenant definition section. The resource definition section defines which resources should be used. There should be at least the resource user and karana, since these are the two core resources. For more complex projects one can also choose other resources related to technical maintenance of karana or pay as you go. A resource definition section for one resource looks like in Listing \ref{listing:configuration.py}.
\lstset{language=Python}  
\begin{lstlisting}[caption={Example for a resource definiton in the \textit{src/configuration.py} file, here for the user resource},label={listing:configuration.py}]
users = 
{
  'metadata': 
  {
    'res_table_id': 1,\
    'schema': 
    {
      'entry_create_schema': "UserSchema",\
      'entry_import_schema': "UserDbSchema"
	},\
    'name': "users",\
    'unique_schema_fields': ['uuid', 'email'],\
    'credentials_login_field': 'email',\
  }\
}
\end{lstlisting}
A resource definition is a python dictionary, holding one key, \textsc{metadata}. The description of each of the subkeys is stored in Table \ref{table:configuration.py} and in the list on \textsc{schema} below.
\begin{table}
\begin{tabular}{ p{3cm}| p{4cm} | p{4cm} }
	key & description & note \\\hline
	 \textsc{res\_table\_id} & the unique id of a resource description & this is an artifact and not used in the code \\\hline
	\textsc{name} & the name of a resource  & resources must be plural \\	\hline
	\textsc{unique} \textsc{\_schema\_fields} & a list of resource attributes which need to be unique  & needs to be a list  \\\hline
	\textsc{credentials\_login} \textsc{\_field} & the  resource attribute  used for login in  & is \textsc{None} if resource is not allowed to login, needs to be in \textsc{unique\_schema\_fields}  \\\hline
\end{tabular}
\caption{Keys of a resource definition}\label{table:configuration.py}
\end{table}
The \textsc{schema} divides into two further keys. Since in the configuration file only meta information is stored one needs to link the resource to the actual model. How to write a model for a resource will be explained later in Subsection \ref{subsection:marshmallow}. Some models are already written, for example the user model. The  \textsc{schema} tells the API which models to used. The models are defined explicitly in the file \textit{src/schema.py}.
\begin{itemize}
	\item \textsc{entry\_create\_schema}: This is the scheme for creation. During creation some attributes are set by the creator (e.g. for user: email, name) and some are created by the system (e.g. for user: uuid, created\_at). So only the presence of the non system created attributes is checked, i.e. is an email an email, is the length of a password sufficient etc.
	\item \textsc{entry\_import\_schema}: This is the scheme for import. When importing a resource to make it accessible it is checked if all entries are present and have the defined properties, e.g. length, is email etc. This is used always when the API is started and is especially important when migrating one data set to another API instance.
\end{itemize}
An example for the tenant definition section is Listing \ref{listing:configuration.py:tenant}
\begin{lstlisting}[caption={Example for a tenant definiton in the \textit{src/configuration.py} file},label={listing:configuration.py:tenant}]
api_metadata = 
{
  'tenant_id': 'SMNTYQIUB4YTC',
  'tenant_customer_name': 'Bintumani e.V.',
  'tenant_login_credentials_resource': 'users',
  'tenant_login_credentials_field': 'credentials', 
  'tenant_used_login_credentials': ['login', 'pwhash'],
  'logging_config_file': 'src/logging.yaml',
  'db_dump': 
  {
    'table_db_path': "storage/table_db.json",\
    'table_meta_db_path': "storage/tablestate_db.json",
    'table_db_folder': "storage/",
  }
}
\end{lstlisting}
\subsection{REST Endpoints: How to use them?}\label{subsection:endpoints}
This subsection deals with the endpoint description. The basic concept is generic. So for each resource there are the same endpoints present. The endpoints are written with Hug as mentioned before and are defined in the file \textit{src/main.py}. This file is also the file which is called when starting the API. \textbf{Note:} one important advantage of using Hug is having the possibility of serving different endpoints. An endpoint version is defined inside the Hug decorator (decorators are Python specific and start with an @). 
\subsubsection{Resource Endpoints}
There are five standard endpoints for each resource. They are described in Table \ref{table:endpoints}. They can be used to get, create change, and delete resources. If you send a body the body needs to be a JSON with one key \textsc{data} holding a string which is a JSON. For example using the HTTP library \href{https://httpie.org/}{httpie}:\\
\begin{tcolorbox}
http post http://localhost:8000/v1/users/new\\
data='\{"name":"Micha","role":"admin","email":"micha@all.de",\\"credentials":\{"login":"micha@all.de","password":"12345"\}\}'
\end{tcolorbox}
\begin{table}
	\begin{tabular}{ p{2cm}| p{4cm} | p{5cm} }
		HTTP Verb & Endpoint & Description \\\hline\hline
		GET & \textsc{v\{nr\}/\{resname\}/\{resid\}} & if \textsc{resid} is not present it returns all resource if it is present it returns this resource \\\hline
		POST & \textsc{v\{nr\}/\{resname\}/new}  & this creates a new resource  \\	\hline
		POST & \textsc{v\{nr\}/\{resname\}/login}  & this generates a token to use the endpoints see also Subsection \ref{subsubsection:auth} \\	\hline
		PUT &\textsc{v\{nr\}/\{resname\}/\{resid\}} & this updates a resource needs at least two changes with given id \\\hline
		PATCH & \textsc{v\{nr\}/\{resname\}/\{resid\}}  & this modifies a resource only one change with given id\\\hline
		DELETE & \textsc{v\{nr\}/\{resname\}/\{resid\}} &  this deletes a resource with given id \\
	\end{tabular}
	\caption{Endpoint description for a resource}\label{table:endpoints}
\end{table}
\subsubsection{Java Web Tokens and Authentication}\label{subsubsection:auth}
The authentication and authorization is managed through \href{https://github.com/jpadilla/pyjwt}{JWT}. If a loginable resource (this is a resource having credentials and a role like the  user resource) once to login it sends its credentials to the appropriate login endpoint (see also Table \ref{table:endpoints}). In return it gets a JSON web token (JWT) holding decrypted information on the creation of the token, which resource id it belongs to, and which role this resource has. This token needs to be send as a bearer in the authorization header of each HTTP request. The Hug framework provides middleware for authentication which decodes the token and confirms that it is valid to use the requested endpoints. The role can limit the number of endpoints, e.g. someone who can see the users may not be allowed to modify them. An example request looks like this:
\begin{tcolorbox}
	http POST http://localhost:8000/v1/users/login user=micha@ex.de password=Mypassword
\end{tcolorbox}
An response would look like this:
\begin{tcolorbox}
HTTP/1.0 200 OK\\
Date: Thu, 02 Mar 2017 16:54:24 GMT\\
Server: WSGIServer/0.2 CPython/3.5.2\\
access-control-allow-headers: content-type\\
access-control-allow-origin: http://localhost:3000\\
content-length: 185\\
content-type: application/json\\

"\{\\
\textbackslash"results\textbackslash": [\\
\{\textbackslash"token\textbackslash": \\
\textbackslash"eyJhbGciOiJIUzI1NiIsInR5cCI6IkpXVCJ9.\\
eyJsb2dpbm5hbWUiOiJtaWNoYUBle\\
C5kZSIsImRhdGEiOnsicm9sZSI6ImFkbWluIn19.\\
Fx9txrY2nBXOKG7BVTYIWEpW2nrPhmjEEu8UUC4TVGE\textbackslash"\}]\\
\}"
\end{tcolorbox}

\subsubsection{Syncing the API }
\subsection{Marshmallow and Model Creation  }\label{subsection:marshmallow}

\subsection{Database: JSON Dumps and GitPython}

\subsection{InfluxDBWrapper and GrafanaWrapper: Configuring Remotely}

\subsection{Py.test and Test Driven Development}

\subsection{JournalCTL and logging}

%----------------------------------------------------------------------------------------
%	BIBLIOGRAPHY
%----------------------------------------------------------------------------------------

\renewcommand{\refname}{\spacedlowsmallcaps{References}} % For modifying the bibliography heading

\bibliographystyle{unsrt}

\bibliography{sample.bib} % The file containing the bibliography

%----------------------------------------------------------------------------------------

\end{document}